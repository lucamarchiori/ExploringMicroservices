\documentclass{beamer}

\usepackage[T1]{fontenc}
\usepackage[utf8]{inputenc}
\usepackage[english]{babel}
\usepackage{lmodern}
\usepackage{listings}
% Use Unipd as theme, with options:
% - pageofpages: define the separation symbol of the footer page of pages (e.g.: of, di, /, default: of)
% - logo: position another logo near the Unipd logo in the title page (e.g. department logo), passing the second logo path as option 
% Use the environment lastframe to add the endframe text
%\usetheme[pageofpages=of, logo=dm_logo.png]{Unipd}
\usetheme[pageofpages=of]{Unipd}

\title{Runtimes for Concurrency and Distribution}
\subtitle{Project: Exploration of orchestration challenges in a microservice-based application}
\author{Luca Marchiori}
\date{June 2024}

\lstdefinestyle{mystyle}{
    basicstyle=\ttfamily\tiny,
    breakatwhitespace=false,         
    breaklines=true,                 
    captionpos=b,                    
    keepspaces=true,                 
    showspaces=false,                
    showstringspaces=false,
    showtabs=false,                  
    tabsize=1,
	aboveskip=5pt, % set the space above the lstlisting block
	belowskip=5pt, % set the space below the lstlisting block
}
\lstset{style=mystyle}


\begin{document}

\maketitle

\begin{frame}{Outline}
	\tableofcontents
\end{frame}


\section{Report corrections}


\begin{frame}{My response to the "B" comments}
	\label{index_1}
	To keep the report within the 10 pages limit, I had to cut some sections relative to theoretical definitions.
		\begin{itemize}
			\item \hyperlink{microservices_definitions}{Microservices definitions}
			\item \hyperlink{rest}{REST architectural style}
			\item \hyperlink{swagger_openapi}{What are Swagger and OpenAPI}
			\item \hyperlink{golang}{Why choosing GO as programming language}
			\item \hyperlink{latest_tag}{Alternative solution to the "latest" tag issue}
			\item \hyperlink{common_ms}{The choice of implementing common and basic microservices}
		\end{itemize}

		Slides are available in the presentation but I prefer to focus on more important topics.
	\end{frame}

	\begin{frame}{My response to the "B" comments}
		\label{index_2}
		The focus of the presentation is on:
			\begin{itemize}
				\item Why virtualization, abstraction and orchestration
				\item How abstraction works in containerization
				\item The notion of business domain separation
				\item The separation between connection state and application state
				\item The problems encountered with the Horizontal Pod Autoscaler
			\end{itemize}
		\end{frame}

\begin{frame}{Missing bibliography}

\end{frame}


\section{Take-home message}
\begin{frame}{Take-home message}
\end{frame}
\section{Learning outcome}
\begin{frame}{Learning outcome}
	\begin{itemize}
		\item \textbf{System design}: 
		\item \textbf{Technologies}: Docker, Kubernetes
	\end{itemize}
\end{frame}
\section{CFU evaluation}
\begin{frame}{CFU evaluation}
	6 CFU equals to 150 hours of work.
	\\
	How I spent my time:
	\begin{itemize}
		\item Lessons: $\sim$ 48 hours
		\item Study: $\sim$ 50 hours
		\item Project: 10 weeks, $\sim$ 6/8 hours per week (60 hours)
	\end{itemize}
	\begin{block}{}
		Good balance between lessons and project, consistent with the CFU assigned.
	\end{block}
\end{frame}

\section{Feedback on the course}
\begin{frame}{Feedback on the course}
	\begin{block}{Good points}
		\begin{itemize}
			\item Very interesting topics that connect basic C.S. concepts with more advanced and real-world applications.
			\item I enjoyed the interactions between the students and the professor, especially for flipped classroom lessons.
			\item It would have been interesting to have more discussion sessions.
		\end{itemize}
	\end{block}
	\begin{block}{Improvements}
		\begin{itemize}
			\item I felt a little lost during the project development. It is hard to start from scratch without a clear idea of what are the expectations.
		\end{itemize}
	\end{block}
\end{frame}


\setbeamercolor{background canvas}{bg=red_unipd}
\begin{frame}{}
	\begin{center}
		\Huge{\textcolor{white}{Thank you for your attention!}}
	\end{center}
\end{frame}

\setbeamercolor{background canvas}{bg=white}
\section{More}
\begin{frame}{Microservices definitions}
	\label{microservices_definitions}
	\begin{block}{}
		“Microservices are \textbf{independently} releasable services that are modeled around a \textbf{business domain}. A service encapsulates functionality and makes it accessible to other services via \textbf{networks}” \footnote{Sam Newman, Building Microservices}
	\end{block}
	\begin{block}{}
		“A microservice is an architectural pattern that arranges an application as a collection of \textbf{loosely coupled}, \textbf{fine-grained services}, communicating through lightweight protocols” \footnote{Martin Fowler, Microservices}
	\end{block}

\hyperlink{index_1}{\beamerbutton{Back to index}}
\end{frame}

\begin{frame}{REST architectural style}
	\begin{block}{}
		REST (Representational State Transfer) is a web architecture style using standard HTTP methods (GET, POST, PUT, DELETE) to interact with resources via URLs.
	\end{block}
	\begin{itemize}
		\item POST: /users (create a new user)
		\item GET: /users/{id} (get a user by id)
		\item GET: /users (get all users)
		\item PUT: /users/{id} (update a user by id)
		\item DELETE: /users/{id} (delete a user by id)
	\end{itemize}
	
	\label{rest}
\hyperlink{index_1}{\beamerbutton{Back to index}}
\end{frame}

\begin{frame}{Why GO as programming language}
	\label{golang}
	\begin{block}{}
		Microservices are language-agnostic, so they can be written in any language as long as they support necessary communication protocols.
		\end{block}
		It is also possible to have different microservices written in different languages as long as they can communicate with each other.
		\newline \newline \newline
		GO was chosen for the project because it has native support for web server making development easier and faster.  \footnote{Thanks to the concurrency features of the language, and the native support for web standards, both Docker and Kubernetes are written in GO.}
\newline
\hyperlink{index_1}{\beamerbutton{Back to index}}
\end{frame}

\begin{frame}{Solution to the "latest" tag issue}
	\label{latest_tag}
	The "latest" tag leads to:
	\begin{itemize}
		\item Unpredictability
		\item Inconsistent updates
		\item Difficult in debugging and replication
	\end{itemize}
	\begin{block}{}
		Instead, use tags consistent with a version control strategy to ensure stability and traceability.
	\end{block}
	This approach allows for precise control over which image version is deployed, facilitating easier rollbacks and consistent environments across development, testing, and production stages.
\hyperlink{index_1}{\beamerbutton{Back to index}}
\end{frame}

\begin{frame}{Common and basic microservices}
	\label{common_ms}
	\begin{block}{}
		Why implementing common and basic microservices such as users, auth, and notifications?
	\end{block}
	\begin{itemize}
		\item The goal was to learn about orchestration, not to develop impressive features.
		\item I didn't want to lose time by thinking of exceptional use cases.
		\item Those are component that I could find in any future project.
	\end{itemize}
\hyperlink{index_1}{\beamerbutton{Back to index}}
\end{frame}

\begin{frame}{What are Swagger and OpenAPI}
	\label{swagger_openapi}
	\begin{block}{Swagger}
		Swagger is a set of open-source tools built around the OpenAPI Specification that can help in design, build, document and consume REST APIs.
\end{block}
\begin{block}{OpenAPI}
		OpenAPI Specification is a specification for building APIs. It is a language-agnostic specification that describes the structure of REST APIs.
\end{block}
In the project i used OpenAPI to define the API of the application and Swagger to have a GUI for reading the API documentation and test the endpoints.
\newline
\hyperlink{index_1}{\beamerbutton{Back to index}}
\end{frame}

\end{document}
